\documentclass{article}
\begin{document}
\begin{center}
{\Large VIPS: Vector-Generating Indoor Positioning System} \\
{\small Jesse Campbell, Roman Parise, Jason Wang, Tanner Emerson, Chandler Ditolla}
\end{center}
\section{Project Description}
In the pre-IoT world, two steps are required to access a device: 1) the location must be determined, and 2) the user must physically interact with the device.
The menu of the IoT device's mobile application eliminates 2).
However, with a sufficient number of IoT devices in a smart home, 1) becomes more complicated in that users are beginning to have to navigate through an ever-increasing number of menus in mobile applications in order to interface with more and more devices.
It would be easier if simply determining the location of a device could allow the user to use the device. \\

We propose a system for determining the location of various IoT devices and being able to remotely access them.
The project has $3$ main parts: \\

\begin{enumerate}
\item To create an indoor positioning system using the Return Time of Flight localization technique in order to create a 3D map of relevant devices via an anchored, central hub. The central hub defines a common origin to which various IoT devices can be referenced and communicates to the IoT devices whether they should turn on, off, or perform more complex functionality. \\
\item To create a module attached to a mobile phone. This module's position and orientation are tracked by sensors in (1). The central hub will use the module's position and orientation to generate a vector extending forward in simulated space to the IoT device at which the user is pointing. \\
\item To create a mobile application that is used to interface with the selected device via the central hub. \\
\end{enumerate}

By taking the user’s orientation into account, we can quickly determine exactly which device a local user is attempting to use and generate the appropriate interface in the mobile application.

\section{Implementation Plan}
Below is a proposed plan for the project's implementation.
\subsection{Stage 0}
Early Deadline: End of this next week \\
Expected Deadline: End of the quarter \\
Late Deadline: First weekend of summer break \\
\begin{itemize}
\item Develop an algorithm to determine the position of IoT devices given distance from various reference nodes. This should include robust error handling
\item Determine the physical mechanism that ascertains the device's distance from these reference nodes (i.e. communication protocol and hardware used)
\item Determine the method by which the user can access the IoT devices, specifically how the user locates the device IoT device
\item Determine the specification of the central hub hardware that contains the reference nodes if one needs to exist
\end{itemize}

\subsection{Stage 1}
Early Deadline: Week 0 of Fall 2018 quarter \\
Expected Deadline: Before fall design review \\
Late Deadline: End of winter break \\
\begin{itemize}
\item Write the program that determines IoT device position using algorithm determined in Stage 0. The input to the program at this stage is simply generated test data
\item Using the hardware and protocol determined in Stage 0, write program (and configure hardware) to acquire the device's distance from a reference node. Data should be verified against hand measurements.
\item Design and assemble the central hub hardware if one needs to exist
\item Design a test UI to interact with a basic IoT device
\end{itemize}

% Testing with basic IoT device
\subsection{Stage 2}
Early Deadline: Before fall design review \\
Expected Deadline: End of winter break \\
Late Deadline: Before winter design review \\
\begin{itemize}
\item Determine a trivially simple IoT device to use for the test
\item Integrate the components developed in Stage 1 into one system
\item Test using the UI to turn on/off the IoT device selected. If central hub is used, the test should pass even in the presence of obstacles between the device and the hub.
\end{itemize}

% Developing a more scalable system
\subsection{Stage 3}
Early Deadline: End of fall quarter \\
Expected Deadline: End of winter break \\
Late Deadline: Before winter design review \\
\begin{itemize}
\item Determine the interface a hardware vendor must provide to the system to utilize the device's functionality
\item Determine how a UI is to be generated from information on IoT device
\item Ensure that the existing protocol supports an arbitrary number of IoT devices. Tweak algorithms if necessary
\end{itemize}

% Designing the more scalable system
\subsection{Stage 4}
Early Deadline: End of winter break \\
Expected Deadline: Before winter design review \\
Late Deadline: N/A \\
\begin{itemize}
\item Write the UI generator. The input is simply descriptions of fictitious IoT devices.
\item Develop the current system so that IoT device information and capabilities are sent to the user
\item Test the current on-off system on multiple trivially simple devices
\end{itemize}

% Testing the more scalable system
\subsection{Stage 5}
Early Deadline: Before winter design review \\
Expected Deadline: N/A \\
Late Deadline: N/A \\
\begin{itemize}
\item Develop a few simple IoT setups with limited functionality. Write the hardware specification for these. (i.e. three LEDs and a choice to turn on one of them)
\item Integrate the system with the UI generator.
\item Test the setup
\end{itemize}

\end{document}
