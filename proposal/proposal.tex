\documentclass{article}
\usepackage{siunitx}
\usepackage{url}
\usepackage{placeins}
\usepackage{csvsimple}
\begin{document}
\begin{center}
{\Large OSTRICH: Organized System Tracking Routes in Consumer Hotspots} \\
{\small Jesse Campbell, Roman Parise, Jason Wang, Tanner Emerson, Chandler Ditolla, Nader Bagherzadeh}
\end{center}
\section{Project Description}
We propose a system for keeping track of the physical location of an outlet or grocery store's inventory and to allow that information to be utilized by consumers and the store itself. \\

For consumers, we plan to create a mobile application that will point them toward the location of the item they seek and potentially find the shortest path to acquire the items in their 'shopping list'. \\

Stores currently employ devices physically attached to a number of expensive products, such as high-end clothing, alcohol, and razor blades. These devices already contain RFID modules in order to alert the store if the stolen good passes through the exit. Our system would replace this as well as provide the store with real-time information on the location of their goods. For inexpensive goods whose cost does not justify security measures, the device could be placed on relevant shelving.\\

This system consists of the following components:
\begin{itemize}
\item A large number of small, inexpensive modules that can be attached to the objects
\item A 'central hub' that serves as the location anchor of our system. This consists of a single processing unit and at least $3$ nodes arrayed across a ceiling in a known configuration. This anchor constantly pings the modules and then uses trilateration to keep track of the coordinates of the modules with respect to itself
\item A simple, mobile application that allows the shopper to acquire the coordinates of the modules
\item A desktop application used by the store to associate the modules with items in the store
\end{itemize}
The modules are to be designed so that they are easy to mass produce to ensure that the system is scalable.
A store employee attaches the module to a product or location of interest and then associates that module with the relevant item in a database of items that store sells using the store's application.
The mapping of these "labels" to the coordinates of the object is stored in a database from the central hub.
The store's application sends this information to a dedicated server that aggregate data from various stores.
The shopper's mobile application can then access the coordinates of any of the given modules by querying a central database that stores the information on the dedicated server.

\section{Implementation Plan}
% What is central hub for proof of concept
A proof-of-concept implementation will include a reasonable number of small modules together with a central hub.
Most of the hardware efforts are dedicated to the design and system-level programming of the small modules.
The initial full-system test will occur in a $100$ft by $100$ft conference room.
The central hub will precisely locate various modules scattered throughout the room. \\

The small modules contain an RF transceiver with an antenna inside a compact housing.
This housing is to be 3D-printed so that the modules can be mass produced.
The housing is to have a mechanical locking mechanism so that it can be connected to an item, such as a shirt.
The store is able to control whether the lock is opened or closed through a software interface. \\

Communication between modules occurs wirelessly at microwave frequencies to ensure that they can be very compact.
The proposed transceiver chip is Decawave's DW1000, which contains fully-integrated RF analog and digital electronics.
The chip is able to handle signals on the $3.5$\si{\giga\hertz} to $6.5$\si{\giga\hertz} band.
Antenna sizes are often on the same order of magnitude as the wavelength of its supported frequencies.
At these frequencies, the antenna size is expected to be on the order of centimeters, meaning that a full communication system can implemented in a very small housing. \\

This particular chip also has the advantage of detecting relatively small objects over long distances.
A communication system using this chip can locate items up to $290$\si{\meter} away to a precision on the order of centimeters.
Thus, the system should have the ability to locate reasonably small locations in a very large store, making it perfect for this application. \\

We have already received several test boards containing DW1000s from the manufacturer.
However, the cost of the board is prohibitive.
Power consumption may also prove to be an issue that limits the system's lifetime.
Thus, we propose designing a custom printed-circuit board (PCB) that removes excess hardware, such as the accelerometer.
This will reduce hardware costs for large volume production.
The software may also need to be adjusted as well to ensure that modules are not unnecessarily pinged and therefore do not consume unnecessary amounts of power.
If the power consumption can be decreased, the battery can be downgraded, also reducing hardware costs. \\

The central hub's software is to periodically request the location of various items in the store.
It communicates this request to the anchor modules.
Those anchors then ping the other modules in the network and acquire the distance they are away from each anchor.
This information is then sent back to the central hub, and the central hub calculates the approximate position of each device based on these distances.
The central hub then uploads this information to a database of the positions of different items in the store.
The main focuses of software development efforts are on the database system and the store and consumer applications.\\

For either of these end users to make sense of the position information, the store is advised to provide a map.
The users and the store are then able to see where each item is located on this map of the store.
Furthermore, the store can select which items are publicly available for end users to see and which are warehouse or secret items that only the store should know. \\

Below is a proposed plan for the project's implementation.
\subsection{Stage 1}
Expected Deadline: Tuesday, November 20th
\begin{itemize}
\item The initial application complete and working with the DW1000 modules
\item Full PCB layout, excluding circuit simulations
\end{itemize}

\subsection{Stage 2}
Expected Deadline: By the End of Fall Quarter \\
\begin{itemize}
\item Store application assuming that we know the battery lifetime of each module, excluding smart algorithm to schedule charging
\item Load firmware onto microcontrollers (MCUs) with programmers. These MCUs will be placed in a socket on the PCB after fabrication.
\item Finish simulations of PCB, order parts for PCB, and place orders for PCB manufacturing and part assembly.
\end{itemize}

\subsection{Stage 3}
Expected Deadline: Before the Start of Winter Quarter \\
\begin{itemize}
\item Acquire access to test equipment for when the boards are finished
\item Determine validation tests to run on the finished boards
\item Design the 3D-printed housing, excluding mechanical stress simulations
\item Develop smart charging algorithm
\item Validate Decawave firmware by testing the MCU with a Decawave chip
\item Begin working on firmware modification to support battery life monitoring
\item Receive the finished boards
\end{itemize}

\subsection{Stage 4}
Expected Deadline: Before Week 2 of Winter Quarter \\
\begin{itemize}
\item Determine any operational faults in the finished boards
\item 3D print the housing
\item Finish firmware modification
\item Improve robustness and security of applications and database
\item Configure optimal algorithm to ping modules
\end{itemize}

\subsection{Stage 5}
Expected Deadline: Before Week 3 of Winter Quarter \\
\begin{itemize}
\item Fix PCB layout if necessary and resubmit for fabrication
\item Otherwise, integrate the boards, 3D-printed housing, firmware, and application-level software
\end{itemize}

\subsection{Stage 6}
Expected Deadline: Before Winter Design Review \\
\begin{itemize}
\item Finish integration and testing of the final system
\end{itemize}

\section{Team Member Responsibilities}
Team member responsibilities are as follows: \\
\begin{itemize}
\item Roman Parise
	\begin{itemize}
	\item Major: Electrical Engineering
	\item Role: Team Captain and System Architecture Lead
	\item Responsibilities: Manages team schedule and defines the system architecture/hardware specification
	\end{itemize}
Personal Statement:

I am Roman Parise, a senior studying Electrical Engineering. I have interned for the past two summers at Intel Corporation, working on datacenter chips, and do research in the Neuromorphic Machine Intelligence Laboratory at UC Irvine. My background in digital circuit design and computer architecture is complemented by my RF/microwave specialization coursework, which includes graduate coursework in microwave circuit design. These experiences have helped me develop a perspective on how electronic systems and wireless communication work at various levels in the design hierarchy.

I want to pursue this research to further the field of indoor positioning and to apply my knowledge of various engineering disciplines to real world problems. Indoor positioning technology has been around for years, but it has never had the impact that this group would like to achieve. My colleagues and I desire to develop a practical and commercializable indoor positioning system that can be easily deployed in existing stores with minimal overhead. UROP's funding is critical to being able to manufacture the hardware to test the system in a real world setting. Working on this project will help me understand how to apply fundamental knowledge to designing consumer products, an integral part of any engineer's career.

\item Jesse Campbell
	\begin{itemize}
	\item Major: Computer Engineering
	\item Role: Application and Database Software Engineer
	\item Responsibilities: Develops the front-end user and store interfaces and the database
	\end{itemize}
Personal Statement:

My name is Jesse Campbell. I am a 4th year transfer student enrolled in the computer engineering major. For my team’s senior design project, we are using RF position tracking inside a grocery store in order to guide customers throughout the store. This idea was founded upon an earlier interest I had in Internet of Things devices that I wrote a research paper on in EECS 1. 

Currently, we have a functioning database to store position and descriptive information of the products we are tracking, a mobile application for customers to use, and a few devices used for tracking that a distributor was willing to give us for free. With additional funding, we will able to purchase more of the devices we already have to perform more comprehensive tests, building to a fully functional product by the end of winter quarter.
\item Chandler Ditolla
	\begin{itemize}
	\item Major: Computer Engineering
	\item Role: System Software Engineer
	\item Responsibilities: Develops firmware for the hardware modules
	\end{itemize}
Personal Statement:

My name is Chandler Ditolla and I'm a 4th year computer engineering student. I am currently a part of a five man team working on our senior capstone project. Our project is a application of rf position tracking inside a grocery store to assist shoppers with finding items in a store. I am interested in this technology because positional tracking in general interests me and I thought this was a novel application doable in the time we have to complete our project. The experience I have in relation to this project is limited to my coding experience cultivated in many of my courses. We have currently accomplished the positional tracking aspect and are close to conducting tests with the user application. The funds received from this proposal will help with pcb design and the purchasing of chips needed for our current system. I believe we should receive funding because we are a hardworking and committed group, in fact our group has been working on our project since the start of summer (the code base) demonstrating our dedication to the project.
\item Tanner Emerson
	\begin{itemize}
	\item Major: Computer Engineering
	\item Role: System Software Engineer
	\item Responsibilities: Develops software interface between the hardware module and the database and aids in the PCB design of the hardware module
	\end{itemize}
Personal Statement:

My name is Tanner Emerson and I am working with a team of Computer/Electrical engineers to implement a real time indoor positioning system in order to help shoppers in a store navigate more efficiently to each item. This means that shoppers would no longer have to wander around aimlessly to find items, leaving more time to spend on the things they choose. This project will allow me to gain invaluable skills with software hardware interfacing. As I move into the workforce, this will be a strong asset, separating me from my peers. I believe that having this experience would have made me go into some of my classes with a better understanding of how to approach a real world problem. 

So far, the team has created an application that allows a user to create a shopping list from items that exist in a specific store. Once they have created a list, we can plot the points on a map showing the user where the items they want are located. Next we will be working on creating our own custom hardware to replace the modules that were purchased from the manufacturer. This will allow us to reduce the price of our final product making it more feasible to implement this on a large scale. We are requesting funding in order to make it possible to make our fulfill our vision of cheaper hardware.
\item Jason Wang
	\begin{itemize}
	\item Major: Electrical Engineering
	\item Role: Hardware Engineer
	\item Responsibilities: Works on PCB design, part selection, and 3D-printed housing for the hardware module
	\end{itemize}
Personal Statement:

My name is Jason Wang. I am a fifth year transfer student pursing my undergraduate education in electrical engineering. My team's goal for the senior design project courses is to design and implement an indoor RF positioning system for a retail environment to guide shoppers to the location of desired products and help employees with inventory management. The hands-on experience I will gain from this project will bolster my understanding of the topics covered in my electronics, digital design, control systems, advanced C programming, electromagnetics, digital signal processing, and antenna design courses. 

So far, my team has made a working Android application for shoppers to interface with the system and implemented a database to store position and product data. We also obtained a small number of RF transceiver development boards from Decawave and are exploring their capabilities. We are seeking funding so we can purchase more development boards to conduct further testing and purchase components to develop hardware more specific to our application.
\end{itemize}

\section{Budget}

The following is the tentative list of parts going on the PCB:

\FloatBarrier

 \begin{table}[h!]
 	\centering
 	\caption{Parts List for PCB}
 	\label{tab:partslist}
 	\csvautotabular{partslist.csv}
\end{table}

\FloatBarrier

PCB assembly and fabrication costs assume that two iterations are done.
Each iteration has five boards manufactured.

\FloatBarrier

\begin{table}[h!]
  	\centering
  	\caption{Itemized Expenditures}
  	\label{tab:costs}
  	\csvautotabular{costs.csv}
\end{table}

\FloatBarrier

\section{References}
\begin{itemize}
\item Decawave's DW1000 Datasheet V2.09 - https://www.decawave.com/sites/default/files/resources/dw1000-datasheet-v2.09.pdf
\item Antenna Pricing - https://www.digikey.com/product-detail/en/taiyo-yuden/AH086M555003-T/587-2204-1-ND/2002902 
\item Battery Pricing - https://www.mouser.com/ProductDetail/TinyCircuits/ASR00036?qs=sGAEpiMZZMt9AXHBRDCfltZePquEfJYbBmgZK9wnboo\%3d
\item DC-DC Converter Pricing - https://www.mouser.com/ProductDetail/TinyCircuits/ASR00036?qs=sGAEpiMZZMt9AXHBRDCfltZePquEfJYbBmgZK9wnboo\%3d
\item DW1000 Pricing - https://www.semiconductorstore.com/cart/pc/viewPrd.asp?idproduct=50015
\item Microcontroller Pricing - https://www.digikey.com/product-detail/en/microchip-technology/ATSAM4N16BA-MUR/1611-ATSAM4N16BA-MURCT-ND/6832601
\item Battery Management IC Pricing - https://www.digikey.com/product-detail/en/texas-instruments/BQ25121AYFPT/296-49478-1-ND/9360696
\item USB Connector Pricing - https://www.digikey.com/product-detail/en/molex-llc/1050170001/WM1399CT-ND/2350885
\item PCB Fabrication Pricing ($5$ $2$-layer boards with size $2$in x $2$in) - https://onequote.sunstone.com/onequote-form/?APIHashKey=AC689457096C621E2335AC01B2FBBCFA
\end{itemize}
\end{document}
